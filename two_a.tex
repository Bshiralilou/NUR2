\section{(2.a)}

In order to generate the fourier plane with the desired symmetry, we have used two matrices, each having 'real' elements that are normaly distributed.\\
We show each complex element of the fourier plane as $a+ib$.\\
The matrix having the values of $a_{kx,ky}$ is thus, symmetric. We can also easily show that the matrix having the values $b_{kx,ky}$ is anti-symmetric.\\
The values of k are in the range $(-512, 512)$. In order to avoid $\sigma = inf$ for k=(0,0), we have assigned a small but nonzero value to k.\\

\lstinputlisting{two_a.py}

\begin{figure}[!htb]
  \centering
  \includegraphics[width=0.7\linewidth]{Plots/fourier_2a_1.png}
  \caption{The gaussian random field with n=-1}
  \label{fig:fig5}
\end{figure}

\begin{figure}[!htb]
  \centering
  \includegraphics[width=0.7\linewidth]{Plots/fourier_2a_2.png}
  \caption{The gaussian random field with n=-2}
  \label{fig:fig6}
\end{figure}

\begin{figure}[!htb]
  \centering
  \includegraphics[width=0.7\linewidth]{Plots/fourier_2a_3.png}
  \caption{The gaussian random field with n=-3}
  \label{fig:fig7}
\end{figure}

In all the plots, the physical size of the axis is in the range (0-512)* scale . Here we have scale = 10 (Mpc).\\
We know from the definition of the Nyquist critical frequency, that, increasing the scale would decrease the critical frequency. This itself means that
 although we are sampling until $k_{max}=512$, we may get the effect of 'aliasing' in our results.
