\section{(3.a)}
In order to solve the system of two ODEs in this question, we have coded the Runge-Kutta method with adaptive step size. In the code, we have defined two numbers N and M which show the number of times where the iteration has continued without and with change in step size, respectively. We also demand that the total number of iterations be < 5000. In this way we can identify whether the solution can be found in a reasonable time or not. We can also see where it gets stuck.\\
The value of the parameters are adopted from the lecture note. Before proceeding, we first derive the differential equation for the growth factor in the Einstein de-Sitter Universe:

\begin{equation}
a(t) = (\frac{3H_{0}t}{2})^{2/3}\,\,\,\,,
\dot{a} = \frac{2}{3}(\frac{3H_{0}}{2})^{2/3} t^{-1/3}
\,\,\,\,, \frac{\dot{a}}{a}=\frac{2}{3t}
\end{equation}
Using these expressions we find:
\begin{equation}
\ddot{D}+\frac{4}{3t} \dot{D} -\frac{2}{3t^{2}} D=0
\end{equation}
Using the anzats of $D\propto t^n$ we can see that this equation has two analytic solutions. One is decaying and the other is growing with $n=1$ and $n=-2/3$.
\lstinputlisting{three_a.py}

\begin{figure}[!htb]
  \centering
  \includegraphics[width=0.7\linewidth]{Plots/ODE_3a_1.png}
  \caption{The solution od ODE for case1}
  \label{fig:fig5}
\end{figure}
\begin{figure}[!htb]
  \centering
  \includegraphics[width=0.7\linewidth]{Plots/ODE_3a_2.png}
  \caption{The solution od ODE for case2}
  \label{fig:fig6}
\end{figure}
\begin{figure}[!htb]
  \centering
  \includegraphics[width=0.7\linewidth]{Plots/ODE_3a_3.png}
  \caption{The solution od ODE for case3}
  \label{fig:fig7}
\end{figure}

\clearpage
In the third case, we have set $h0=1$ which is somehow large. The reason for this choice was that for smaller values of $h0$, the iteration could not proceed even one step further. This could be because the equation is stiff around $t=0$. By choosing large initial step size, we 'pass' this region with the cost of not getting accurate answer.
